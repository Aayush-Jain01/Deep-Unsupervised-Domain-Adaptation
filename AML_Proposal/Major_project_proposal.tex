\documentclass[12pt]{extarticle}
\usepackage[utf8]{inputenc}
\usepackage{cite}
\usepackage{graphicx}
\usepackage[margin=0.8in,top=1cm,bottom=0.5in]{geometry}
\title{Advanced Machine Learning \\ Project Proposal}
\author{Venkata Santosh Sai Ramireddy Muthireddy \\ \newline Alan Preciado Grijalva}
\date{February 2020}

\begin{document}

\maketitle

In this project we will study some domain adaptation techniques being used in deep neural networks. 

It has been well proved that deep networks are efficient at extracting features from a given (source) labeled dataset. However, it is not always the case that they can generalize well to other (target) datasets which very often have a different underlying distribution.

An example of the limitations of a model is its lack of ability to classify objects in images with features it has never been trained with (e.g. different backgrounds, orientation and context in general). To overcome this, domain adaptation algorithms attempt to transfer invariant features from one data domain to another. 

In the context of neural networks, Deep CORAL \cite{sun2016deep} is an unsupervised convolution neural network  model that compensates for decrease in performance when there is a shift in domain by aligning the second-order-statistics (correlation of output layer activation) of source and target distributions. The authors introduce an end-to-end domain adaptation work flow using a loss function that consists of a classification loss and a “coral loss” that learns invariant features.

Another unsupervised learning model for domain adaptation is PixDA \cite{bousmalis2017unsupervised}. In this work, the authors map representations from one domain to another at the pixel level using generative adversarial networks. The paper reports an improvement in domain adaptation (classification tasks) from MNIST to MNIST-M of more than 30 \% over Deep CORAL and other SOTA methods.

\section{Project Proposals}
Our proposal consists in three possible avenues to work on domain adaptation using neural networks. 
\begin{enumerate}

\item Implementation and evaluation of multiple domain adaptation algorithms (e.g. deep CORAL, PixelDA) on the “office” dataset for classification tasks. 
    \item Implementation of a single domain adaptation algorithm and evaluate its classification performance in multiple datasets like both source and target datasets being either real or synthetic.
    \item Implementing Deep CORAL model on a particular dataset (farm animal pictures) to do domain adaptation of pose estimation data. This data is in the context of research carried in Fraunhofer FIT.
\end{enumerate}
\bibliographystyle{plain}
\bibliography{M335}

\end{document}
